\chapter{Evoluční Algoritmy}

\section{Zakládní definice}
V této sekci bychom chtěli definovat některé užitečné pojmy a zároveň vysvětlit, jakou hrají roli při návrhu a běhu evolučního algoritmu.
\subsection{Gen}
Gen představuje jedno řešení pro daný problém. Je důležité zvolit vhodné kódování tohoto řešení. Holland si ve svém původním návrhu představoval, že každé řešení bude binárně kódováno, ale později se ukázalo, že je možné dosáhnout lepších výsledků, když kódováním reprezentujeme boilding-blocks daného řešení.

\subsection{Populace a Generace}
Populace je neuspořádáná množina genů. Populace se ze začatku běhu algoritmu většinou naplní náhodnými geny a postupnout aplikací evolučních operátorů se v ní vyvíjí lepší a lepší řešení. Generace představuje stav populace v konkrétním čase.

\subsection{Evoluční operátory}
Evoluční operátory jsou funkce, které můžeme aplikovat na celou popuplaci a neviiim, explorace exploatace.

\subsection{Křížení}

\subsection{Mutace}

\subsection{Selekce}

\subsection{Fitness}

\section{Přiklady}
\subsection{Problém Batohu}
