\chapter*{Úvod}
\addcontentsline{toc}{chapter}{Úvod}

V posledních dekádách se umělá inteligence stala klíčovým prvkem v mnoha technologických a vědeckých oborech, včetně počítačových her \cite{mnih2013playing} \cite{Vinyals2019}. Jedním z~fascinujících přístupů, jak umělá inteligence může ovlivnit interakci s počítačovými hrami, je použití evolučních algoritmů. Tyto algoritmy, inspirované biologickou evolucí, nabízejí zajímavý mechanismus pro automatizované řešení problémů a optimalizaci procesů \cite{EibenSmith2015}.

Poly Bridge je počítačová hra zaměřená na stavební inženýrství a řešení fyzikálních hádanek. Hra žádá od hráčů, aby stavěli mosty přes různé rozpětí vodních ploch s omezenými zdroji a za dodržení specifických parametrů \cite{drycactus}. Cílem této bakalářské práce je prozkoumat, jak mohou být evoluční algoritmy použity pro automatické generování a optimalizaci mostních konstrukcí v rámci hry.

Cílem této bakalářské práce je nejen prozkoumat teoretické aspekty evolučních algoritmů, ale především demonstrovat jejich praktické využití a potenciál v~kontextu moderních počítačových her. Tato práce tedy může sloužit jako příklad, jak moderní metody umělé inteligence mohou překonat tradiční přístupy používané komunitou hráčů počítačových her. 

Práce nejprve představí základní koncept evolučních algoritmů a vysvětlí jejich principy a metody. Následně bude analyzovat specifika hry Poly Bridge, aby bylo možné vytvořit fyzikální simulované prostředí pro aplikaci těchto algoritmů. Praktická část práce se zaměří na implementaci vybraných algoritmů, jejich testování a hodnocení na základě efektivity a práci se zdroji.


