%%% Fiktivní kapitola s ukázkami citací

\chapter{Odkazy na literaturu}

Při zpracování bibliografie (přehledu použitých zdrojů) se řídíme
normou ISO 690 a zvyklostmi oboru. V~\LaTeX{}u nám pomohou balíčky
\textsf{biblatex}, \textsf{biblatex-iso690}.
Zdroje definujeme v~souboru \texttt{literatura.bib} a pak se na ně v~textu
práce odkazujeme pomocí makra \verb|\cite|. Tím vznikne odkaz v~textu
a~odpovídající položka v~seznamu literatury.

V~matematickém textu obvykle odkazy sázíme ve tvaru
\uv{Jméno autora/autorů [číslo odkazu]}, případně
\uv{Jméno autora/autorů (rok vydání)}.
V~českém/slovenském textu je potřeba se navíc vypořádat
s~nutností skloňovat jméno autora, respektive přechylovat jméno
autorky.
K~doplňování jmen se hodí příkazy \verb|\citet|, \verb|\citep|
z~balíčku \textsf{natbib}, ale je třeba mít na paměti, že
produkují referenci se jménem autora/autorů v~prvním pádě a~jména
autorek jsou nepřechýlena.

Jména časopisů lze uvádět zkráceně, ale pouze v~kodifikované podobě.

Při citování je třeba se vyhnout neověřitelným, nedohledatelným a nestálým zdrojům.
Doporučuje se pokud možno necitovat osobní sdělení, náhodně nalezené webové stránky,
poznámky k přednáškám apod. Citování spolehlivých elektronických zdrojů (maji ISSN
nebo DOI) a webových stránek oficiálních instituci je zcela v pořádku. Citujeme-li
elektronické zdroje, je třeba uvést URL, na němž se zdroj nachází, a~datum přístupu
ke zdroji.

\section{Několik ukázek}

Aktuální verzi této šablony najdete v~gitovém repozitáři \cite{ThesisTemplate}.
Také se může hodit prohlédnout si další návody udržované Martinem Marešem
\cite{ThesisWeb}.

Mezi nejvíce citované statistické články patří práce Kaplana a~Meiera a~Coxe
\cite{KaplanMeier58, Cox72}. \citet{Student08} napsal článek o~t-testu.

Prof. Anděl je autorem učebnice matematické statistiky \cite{Andel98}.
Teorii odhadu se věnuje práce \citet{LehmannCasella98}.
V~případě odkazů na specifickou informaci
(definice, důkaz, \dots) uvedenou v~knize bývá užitečné uvést
specificky číslo kapitoly, číslo věty atp. obsahující požadovanou
informaci, např. viz \citet[Věta 4.22]{Andel07}.

Mnoho článků je výsledkem spolupráce celé řady osob. Při odkazování
v~textu na článek se třemi autory obvykle při prvním výskytu uvedeme
plný seznam: \citet*{DempsterLairdRubin77} představili koncept EM
algoritmu. Respektive: Koncept EM algoritmu byl představen v~práci
Dempstera, Lairdové a~Rubina~\cite{DempsterLairdRubin77}. Při každém
dalším výskytu již používáme zkrácenou verzi:
\citet{DempsterLairdRubin77} nabízejí též několik příkladů použití EM
algoritmu. Respektive: Několik příkladů použití EM algoritmu lze
nalézt též v~práci Dempstera a~kol.~\cite{DempsterLairdRubin77}.

U~článku s~více než třemi autory odkazujeme vždy zkrácenou formou:
První výsledky projektu ACCEPT jsou uvedeny v~práci Genbergové a~kol.~\cite{Genberget08}.
V~textu \emph{nenapíšeme:} První výsledky
projektu ACCEPT jsou uvedeny v~práci \citet*{Genberget08}.
