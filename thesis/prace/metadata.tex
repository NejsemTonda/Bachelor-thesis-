%%% Vyplňte prosím základní údaje o závěrečné práci.
%%% Automaticky se pak vloží na všechna místa, kde jsou potřeba.

% Druh práce:
%	"bc" pro bakalářskou
%	"mgr" pro diplomovou
%	"phd" pro disertační
%	"rig" pro rigorozní
\def\ThesisType{bc}

% Název práce v jazyce práce (přesně podle zadání)
\def\ThesisTitle{Evoluční algoritmy pro konstrukci 2D mostů ve hře Poly Bridge}

% Název práce v angličtině
\def\ThesisTitleEN{Evolutionary algorithms for 2D bridge construction in the Poly Bridge game}

% Jméno autora (vy)
\def\ThesisAuthor{Václav Krňák}

% Rok odevzdání
\def\YearSubmitted{2024}

% Název katedry nebo ústavu, kde byla práce oficiálně zadána
% (dle Organizační struktury MFF UK:
% https://www.mff.cuni.cz/cs/fakulta/organizacni-struktura,
% případně plný název pracoviště mimo MFF)
\def\Department{Katedra teoretické informatiky a matematické logiky}
\def\DepartmentEN{Department of Theoretical Computer Science and Mathematical Logic}

% Jedná se o katedru (department) nebo o ústav (institute)?
\def\DeptType{\xxx{Department}}
\def\DeptTypeEN{\xxx{Department}}

% Vedoucí práce: Jméno a příjmení s~tituly
\def\Supervisor{Mgr. Roman Neruda, CSc.}

% Pracoviště vedoucího (opět dle Organizační struktury MFF)
\def\SupervisorsDepartment{Katedra teoretické informatiky a matematické logiky}
\def\SupervisorsDepartmentEN{Department of Theoretical Computer Science and Mathematical Logic}

% Studijní program (kromě rigorozních prací)
\def\StudyProgramme{Informatika}

% Nepovinné poděkování (vedoucímu práce, konzultantovi, tomu, kdo
% vám nosil pizzu a vařil čaj apod.)
\def\Dedication{%
\xxx{Poděkování.}
}

% Abstrakt (doporučený rozsah cca 80-200 slov; nejedná se o zadání práce)
\def\Abstract{%
Tato bakalářská práce se věnuje evolučním algoritmům a jejich aplikaci a řešení několika úrovní ze polybridge. Polybridge je logická hra se sandboxovím prostředím , ve kterém je hráč veden k tomu aby postavil 2D mostní konstrukci podobnou příhradovému mostu. Hráč musí dbát na stabilitu této konstrukce, ale zároveň i na materálovou náročnost. Vzhledem k složitosti a variabilitě problémů, které hra představuje, jsme se rozhodli pro použití evolučních algoritmů, které poskytují robustní a flexibilní nástroj pro optimalizaci a automatický design. Navrhli jsem několik různých genetických operátorů a jejich účinnost mezi sebou porovnali.
} 

% Anglická verze abstraktu
\def\AbstractEN{%
\xxx{Abstract.}
}

% 3 až 5 klíčových slov (doporučeno) oddělených \sep
% Hodí se pro nalezení práce podle tématu.
\def\ThesisKeywords{%
Evoluční algoritmy\sep
Polybridge\sep
Optimalizace\sep
}

\def\ThesisKeywordsEN{
Evolutionary Algorithms\sep
Polybridge\sep
Optimalization\sep
}

% Pokud některá z položek metadat obsahuje TeXové řídící sekvence, je potřeba
% dodat i verzi v obyčejném textu, která se objeví v metadatech formátu XMP
% zabudovaných do výstupního souboru PDF. Pokud si nejste jistí, prohlédněte si
% vygenerovaný soubor thesis.xmpdata.
\def\ThesisAuthorXMP{\ThesisAuthor}
\def\ThesisTitleXMP{\ThesisTitle}
\def\ThesisKeywordsXMP{\ThesisKeywords}
\def\AbstractXMP{\Abstract}

% Máte-li dlouhý abstrakt a nechceme se mu vejít na informační stranu,
% můžete použít toto nastavení ke zmenšení písma informační strany.
% (Uvažte nicméně zkrácení abstraktu, to je často lepší.)
\def\InfoPageFont{}
%\def\InfoPageFont{\small}  % odkomentujte pro zmenšení písma
