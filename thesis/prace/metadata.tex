%%% Vyplňte prosím základní údaje o závěrečné práci.
%%% Automaticky se pak vloží na všechna místa, kde jsou potřeba.

% Druh práce:
%	"bc" pro bakalářskou
%	"mgr" pro diplomovou
%	"phd" pro disertační
%	"rig" pro rigorozní
\def\ThesisType{bc}

% Název práce v jazyce práce (přesně podle zadání)
\def\ThesisTitle{Evoluční algoritmy pro konstrukci 2D mostů ve hře Poly Bridge}

% Název práce v angličtině
\def\ThesisTitleEN{Evolutionary algorithms for 2D bridge construction in the Poly Bridge game}

% Jméno autora (vy)
\def\ThesisAuthor{Václav Krňák}

% Rok odevzdání
\def\YearSubmitted{2024}

% Název katedry nebo ústavu, kde byla práce oficiálně zadána
% (dle Organizační struktury MFF UK:
% https://www.mff.cuni.cz/cs/fakulta/organizacni-struktura,
% případně plný název pracoviště mimo MFF)
\def\Department{Katedra teoretické informatiky a matematické logiky}
\def\DepartmentEN{Department of Theoretical Computer Science and Mathematical Logic}

% Jedná se o katedru (department) nebo o ústav (institute)?
\def\DeptType{Katedra}
\def\DeptTypeEN{Department}

% Vedoucí práce: Jméno a příjmení s~tituly
\def\Supervisor{Mgr. Roman Neruda, CSc.}

% Pracoviště vedoucího (opět dle Organizační struktury MFF)
\def\SupervisorsDepartment{Katedra teoretické informatiky a matematické logiky}
\def\SupervisorsDepartmentEN{Department of Theoretical Computer Science and Mathematical Logic}

% Studijní program (kromě rigorozních prací)
\def\StudyProgramme{Informatika}

% Nepovinné poděkování (vedoucímu práce, konzultantovi, tomu, kdo
% vám nosil pizzu a vařil čaj apod.)
\def\Dedication{%
\xxx{Poděkování.}
}

% Abstrakt (doporučený rozsah cca 80-200 slov; nejedná se o zadání práce)
\def\Abstract{%
V této bakalářské práci se budeme zabývat řešením několika úrovní ze hry Poly Bridge pomocí umělé inteligence. Poly bridge je logická hra se sandboxovým prostředím, ve kterém je hráč veden k tomu aby postavil 2D mostní konstrukci podobnou příhradovému mostu. Hráč musí dbát na stabilitu této konstrukce, ale zároveň i na materiálovou náročnost. Vzhledem k složitosti a variabilitě problémů, které hra představuje, jsme se rozhodli použít evoluční algoritmy. Cílem práce je návrh několika genetický operátorů, které optimalizují mostní strukturu, jejich porovnání a využití mostů, které produkují, ve hře Poly Bridge.
} 

% Anglická verze abstraktu
\def\AbstractEN{%
This bachelor thesis proposes an artificial intelligence to solve several levels from the game Poly Bridge. Poly bridge is a puzzle game with a sandbox environment in which the player is led to build a 2D bridge structure similar to a truss bridge. The player has to pay attention to the stability of this structure, but also to the material requirements. Given the complexity and variability of the problems posed by the game, we decided to use evolutionary algorithms. The aim of this thesis is to design several genetic operators that optimize the bridge structure, compare them and use the bridges they produce in the game Poly Bridge. 
}

% 3 až 5 klíčových slov (doporučeno) oddělených \sep
% Hodí se pro nalezení práce podle tématu.
\def\ThesisKeywords{%
Evoluční algoritmy\sep
Polybridge\sep
Optimalizace\sep
}

\def\ThesisKeywordsEN{
Evolutionary Algorithms\sep
Polybridge\sep
Optimalization\sep
}

% Pokud některá z položek metadat obsahuje TeXové řídící sekvence, je potřeba
% dodat i verzi v obyčejném textu, která se objeví v metadatech formátu XMP
% zabudovaných do výstupního souboru PDF. Pokud si nejste jistí, prohlédněte si
% vygenerovaný soubor thesis.xmpdata.
\def\ThesisAuthorXMP{\ThesisAuthor}
\def\ThesisTitleXMP{\ThesisTitle}
\def\ThesisKeywordsXMP{\ThesisKeywords}
\def\AbstractXMP{\Abstract}

% Máte-li dlouhý abstrakt a nechceme se mu vejít na informační stranu,
% můžete použít toto nastavení ke zmenšení písma informační strany.
% (Uvažte nicméně zkrácení abstraktu, to je často lepší.)
\def\InfoPageFont{}
%\def\InfoPageFont{\small}  % odkomentujte pro zmenšení písma
