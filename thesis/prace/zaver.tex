\chapter{Závěr}

\addcontentsline{toc}{chapter}{Závěr}


V této bakalářské práci jsme zkoumali možnosti využití evolučních algoritmů pro konstrukci mostů ve hře Poly Bridge. Práce ukázala teoretický základ evolučních algoritmů a aplikuje tyto principy na specifické problémy hry. Bylo vytvořeno fyzikální prostředí schopné simulovat chování mostů. Navrhli jsme a implementovány různé genetické operátory, které byly následně testovány pro tvorbu a optimalizaci mostových struktur.

Experimentální výsledky ukázaly, že přístup založený na evolučních algoritmech je schopen efektivně generovat funkční mostní konstrukce, které nejenže splňují požadavky dané úrovně hry, ale často jsou nákladově efektivnější ve srovnání s konstrukcemi vytvořenými člověkem. Bohužel, kvůli naší nepřesné simulaci fyzkálního prostředí nebylo možné použít všechna řešení i ve hře.

Existuje několik možností, jak bychom mohli vylepšit celkové výsledky našich experimentů. Jednou z cest je zdokonalení fyzikální simulace, kterou využíváme. Toho bychom mohli dosáhnout například rozšířením a zpřesněním sady testů nebo experimentováním s různými komponentami enginu Box2D.

Také by bylo užitečné prozkoumat další metody umělé inteligence, jako je zpětnovazební učení, neuronové sítě nebo techniku zvanou \emph{novelty search} \cite{lehman2011}, které by mohly přinést nové perspektivy a vylepšení našeho výzkumu. 

