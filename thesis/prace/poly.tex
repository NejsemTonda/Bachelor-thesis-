\chapter{Hra PolyBridge}

Poly Bridge je logická simulovaná hra, která byla vyvinuta nezávislým vývojářským týmem Dry Cactus. Hráči mají za úkol navrhovat mosty, které musí odolat zátěži projíždějících vozidel a zároveň splňovat omezení daná rozpočtem a dostupnými materiály. Díky svému základu v realistické fyzice nabízí Poly Bridge možnost experimentovat s různými stavebními technikami a materiály.

Poly Bridge odráží řadu reálných principů stavitelství mostů. Stejně jako ve skutečnosti, mosty ve hře vyžadují pečlivé plánování a zohlednění sil, které na ně působí. Tyto síly zahrnují napětí, tlak, ohyb a torzi. Hra tedy poskytuje hráčům intuitivní pochopení těchto principů prostřednictvím připravených výzev a následné simulace. K tomu hráči využívají různé materiály jako jsou dřevo, ocel a lano, přičemž každý materiál má specifické vlastnosti a omezení vzhledem k ceně a nosnosti.

Zajímavým aspektem hry je také komunitní prvek. Hráči mohou sdílet své vlastní návrhy mostů a kampaní online, kde je mohou ostatní hráči testovat a hodnoti. Díky tomu, je možné sdílet techniky a nápady mezi hráči z celého světa. 

"Popusťte uzdu své inženýrské kreativitě v poutavém a neotřelém simulátoru stavby mostů se všemi zvonky a píšťalkami. Užijte si hodiny řešení fyzikálních hádanek v kampani a pak se vrhněte do sandboxu a vytvářejte vlastní návrhy mostů a hádanky!" (přeloženo Dry Cactus citace)

\section{Vlastní hra}

jak se hra hraje, jak funguje přidávání přidávání materiálu, jaké tehcniky se často používají

\section{Kampaně}

Jak je hra organizovaná, počet kapamní, počet úrovní, featury.
Zmíňka o tom, že budeme experimenty měřit na prvních 5 úrovních ze hry
