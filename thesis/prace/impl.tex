\chapter{Implementace}

\xxx{Co je těžké na udělání kopie hry (stejná fyzika)}

\xxx{Už se o to někdo pokoušel (video z youtube asi?)}

\xxx{zmíňka o publikací, kde se staví mosty pomocí EA ale ne mosty do polybridge}


\section{Fyzikální engine}

Jako fyzikální engine pro simulaci jsme zvolili Box2D (citace Box2D). Box2D je open-source fyzikální engine, který poskytuje simulaci pohybu objektů ve 2D prostoru. Je často využíván ve vývoji počítačových her ale také simulací a umožňuje snadné zpracování kolizí, gravitace, tuhosti objektů a dalších fyzikálních jevů. Tento engine byl původně implementován v jazyce C++, avšak díky dostupným knihovnám jej můžeme používat Pythonu, což zvyšuje jeho rychlost a umožňuje nám iterovat přes rozsáhlé množství simulací.

\section{Aproximace hře Poly Bridge}

V naší simulaci jsme implementovali různé aspekty hry pomocí následujících komponent knihovny Box2D:

\begin{itemize}
    \item \textbf{Materiály}: Ty jsou modelovány jako dynamické objekty, pro které používáme `Box2D.b2DynamicBody`.
    \item \textbf{Klouby}: Pro spojení různých materiálů jsme využili `Box2D.b2RevoluteJoint`.
    \item \textbf{Zátež na prvky}: Abychom zjistili síly působící na jednotlivé elementy v simulaci, používáme metodu `b2body.GetReactionForce()`, která vrací reakční sílu vzniklou v důsledku interakcí těles.
\end{itemize}

\subsection{Testy}

Abychom v naší simulaci co nejvěrněji napodobyli chování fyzikálních prvků jako ve hře polybridge, zavedli jsme šest různorodých testů. Tyto testy zkoumají aspekty fyzikální simulace, jako jsou odolnost materiálů v proměnlivých podmínkách, hmotnost materiálů a interakci sil mezi objekty.

Zvolené testy jsou následující:

\begin{itemize}
    \item \textbf{Dvě vozovky mezi dvěma pevnými body} Testujeme, zda konstrukce praskne pod zatížením, což by mělo být očekávaným výsledkem.
    \item \textbf{Šest dřevěných dílů mezi dvěma pevnými body} Očekáváme, že konstrukce vydrží bez prasknutí.
    \item \textbf{Sedm dřevěných dílů mezi dvěma pevnými body} V tomto testu očekáváme, že konstrukce pod tíhou praskne.
    \item \textbf{Symetrický obrazec z 13.66 metrů vozovky, zavěšený na jednom kusu vozovky} Testujeme, zda vozovka unese zatížení bez prasknutí.
    \item \textbf{Symetrický obrazec z 14.66 metrů vozovky, zavěšený na jednom kusu vozovky} V tomto případě testujeme, zda konstrukce nevydrží zatížení a praskne.
    \item \textbf{Komplexní most z vozovek a dřeva, po kterém přejede auto} Cílem tohoto testu je prozkoumat interakci sil mezi různými materiály, kdy očekáváme, že most vydrží přejetí auta.
\end{itemize}

Vizualizaci testů můžeme vidět na obrázku 2 a 3

\begin{figure}[ht]
    \centering
    \begin{minipage}{0.49\textwidth}
        \centering
        \includegraphics[width=\linewidth]{img/poly_tests.png}
    \end{minipage}\hfill
    \begin{minipage}{0.49\textwidth}
        \centering
        \includegraphics[width=\linewidth]{img/sim_tests.png}
    \end{minipage}
    \caption{Vizualizace testů ve hře polybridge (nalevo) a v simulaci (vpravo)}
    \label{fig:1}
\end{figure}


\xxx{simulace má 4 různé hyperparametry parametry, udělali jsme random search abychom splniny co nejvíce testů a aby simulace byla co nejstabilnější}

\subsection{Úrovně}

\xxx{Jak jsem úrovně z polybridge udělal ve svojí simulaci. Možná obrázky z polybridge a z mojí simulace}

\begin{figure}[ht]
    \centering
    \begin{minipage}{0.49\textwidth}
        \centering
        \includegraphics[width=\linewidth]{img/poly_lvl1.png}
    \end{minipage}\hfill
    \begin{minipage}{0.49\textwidth}
        \centering
        \includegraphics[width=\linewidth]{img/impl_lvl1.png}
    \end{minipage}
    \caption{Vizualizace první úrovně ve hře polybridge (nalevo) a v simulaci (vpravo)}
    \label{fig:2}
\end{figure}

\begin{figure}[ht]
    \centering
    \begin{minipage}{0.49\textwidth}
        \centering
        \includegraphics[width=\linewidth]{img/poly_lvl2.png}
    \end{minipage}\hfill
    \begin{minipage}{0.49\textwidth}
        \centering
        \includegraphics[width=\linewidth]{img/impl_lvl2.png}
    \end{minipage}
    \caption{Vizualizace druhé úrovně ve hře polybridge (nalevo) a v simulaci (vpravo)}
    \label{fig:3}
\end{figure}

\begin{figure}[ht]
    \centering
    \begin{minipage}{0.49\textwidth}
        \centering
        \includegraphics[width=\linewidth]{img/poly_lvl3.png}
    \end{minipage}\hfill
    \begin{minipage}{0.49\textwidth}
        \centering
        \includegraphics[width=\linewidth]{img/impl_lvl3.png}
    \end{minipage}
    \caption{Vizualizace třetí úrovně ve hře polybridge (nalevo) a v simulaci (vpravo)}
    \label{fig:4}
\end{figure}

\begin{figure}[ht]
    \centering
    \begin{minipage}{0.49\textwidth}
        \centering
        \includegraphics[width=\linewidth]{img/poly_lvl4.png}
    \end{minipage}\hfill
    \begin{minipage}{0.49\textwidth}
        \centering
        \includegraphics[width=\linewidth]{img/impl_lvl4.png}
    \end{minipage}
    \caption{Vizualizace čtvrté úrovně ve hře polybridge (nalevo) a v simulaci (vpravo)}
    \label{fig:5}
\end{figure}


\section{Aplikace evolučních algoritmů}

\xxx{Jak jsem udělal EA}

\xxx{můžů/mám sem dávat kousky kódu, z pythoní implementace? to by bylo asi nejednodušší}

\subsection{Jednoduchý návrh}

\xxx{reprezentace je vektor dvojic čísel, každá dvojce říká, kde bude další kliknutí myši}
\xxx{turnajová selekce, jednoboidové křížení, mutace náhodně změní číslo +/- 1}

\xxx{Několik důvodů, proč si myslím, že to vůbec nevyšlo a jak je plánuju opravit}

Jak můžeme vidět v experimentu (odkaz experiment) algortimu se nedaří stavit příliš kvalitní mosty. Domníváme se, že by to mohlo být z následujících důvodů.

\begin{itemize}
    \item Z pricipu reprezentace jedince je nepravděpodobné, aby vznikaly krátké hrany, které mohou být klíčové pro kaviltní řešení.
    \item Malá mutace na začátku genu může mít velký vliva na celkouvou strukturu mostu.
    \item Krížení v naší reprezentaci nedává smysl.
    \item Fitness funkce nevrací dobrou zpětnou vzdbu o kvalitě jedince \xxx{obrázek s porvnáním?}
\end{itemize}

\subsection{Polární kódování}

\xxx{reprezentace podobně jako v přechozím případě, ale dvojce čísel jsou délka nově přidaného materiálu a ůhel}
\xxx{jinak stejně jako u předchozího případu}

\subsection{Vylepšená fitness funkce}

\xxx{Fitness nám nedává moc informací o tom, jak je most dobrý. Přidali jsem několik další komponentů do fitness (proč by se to mělo zlepšit? citace knížka)}

\begin{itemize}
    \item Penalizace za umisťování materiálů, který se nespojí s další materíálem. Lépe propojený most by měl mít lepší stabilitu
    \item Penalizace za všechny kotvy, které jedinece nepoužil v podobě vzdálenosti každého kliknutí ke všem nepoužitým kotvám. Most který používá více kotev by měl být stabilnější
    \item Penalizace za předčasně ukončenou simulaci. Simulace se předčasně ukončí pokud auto spadne, nebo pokud se dlouho nepohybuje
\end{itemize}

\subsection{Měnící se fitness}

\xxx{Udělám problém jednodušší. Dám břehy blíž k sobě, jakmile bude průměrná fitness dostatečně nízko, břehy oddálím}

\subsection{Grafové kódování}

\xxx{Grafové kódování. Ukládám zvlášť vrcholy a hrany mostu}
\xxx{Popsané jak budou pak vypadat mutace a křížení}

\subsection{Lepší inicializace}

\xxx{Udělám stejnou init jako v článku (citace mosty)}
