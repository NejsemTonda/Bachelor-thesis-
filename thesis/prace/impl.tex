\chapter{Implementace}

Co je těžké na udělání kopie hry (stejná fyzika)

Už se o to někdo pokoušel (video z youtube asi?)

zmíňka o publikací, kde se staví mosty pomocí EA ale ne mosty do polybridge


\section{Fyzikální engine}

Krátký odstaveček o tom, jak jsem použil Box2D (citace Box2D). Co to je a proč jsem si to vybral

\section{Aproximace hře Poly Bridge}

Jak jsem udělal to že se věci v mojí simulaci chovají stejně jako ve hře.
Jaké data jsem dostal ze hry. 

\subsection{Testy}

Jaké testy jsem udělal, abych věděl jako moc se hře podobám. 

\Subsection{Úrovně}

Jak jsem úrovně z polybridge udělal ve svojí simulaci. Možná obrázky z polybridge a z mojí simulace

\section{Aplikace evolučních algoritmů}

Jak jsem udělal EA
\subsection{Jednoduchý návrh}

Ukázaka

Několik důvodů, proč si myslím, že to vůbec nevyšlo a jak je plánuju opravit

\subsection{Polární kódování}
Předchozí experiment dělal jenom dlouhé příčky, takhle budu moc udělat i krátkou. Pomůže to vůbec?

\subsection{Vylepšená fitness funkce}

Fitness nám nedává moc informací o tom, jak je most dobrý. Přidali jsem několik další komponentů do fitness (proč by se to mělo zlepšit? citace knížka)

\subsection{Měnící se fitness}

Zlepším se tím, že udělám problém jednodušší. Postupně budu stěžovat dokud nezkonverguju

\subsection{Křížení}

Dává v mém kódování křížení smysl? Má moc velký vliv na celou strukturu

\subsection{Grafové kódování}

Změním kódování aby dávalo smysl. Mutace a Křížení pak budou reálně k něčemu a nebudu si to kazit

\subsection{Lepší inicializace}

Udělám stejnou init jako v jiném článku (citace mosty). Bude mít pak křízení smysl? exp
