\chapter{Implementace}

V této části bakalářské práce se zaměříme na praktickou implementaci evolučních algoritmů. Kromě toho je nutné ale i implementovat fyzikální prostředí, podobné tomu jako je ve hře polybridge. V první řadě se zaměříme aby se naše simulace co nejvěrněji podobala hře, což umožní použít řešení navrhnuté evolučními algoritmy i ve hře. Následně navrhneme několik různých podob evolučního algoritmu pro stavbu mostu a ty mezi sebou porovnáme. Jako programovací jazyk jsme zvolili Python. Kompletní implementaci i se stručnou dokumentací můžeme nalézt na githubu \cite{git}.

\section{Související literatura}

Pro návrh evolučních operátorů využijeme inspiraci z existujících studií, které se zaměřili na podobný problém. Konkrétně z diplomové práce Huga Lispectora \cite{Lispector2022} adoptujeme metodu chytré inicializace mostů. Dále použijeme některé evoluční operátory z programovacího projektu autora AstroSam \cite{AstroSam2023}. Je třeba poznamenat, že obě zmíněné práce se nevěnují přesně stejnému problému, což nám znemožňuje přímé porovnání našich výsledků s těmito studiemi.

\section{Fyzikální engine}

Jako fyzikální engine pro simulaci jsme zvolili Box2D \cite{box2d}. Box2D je open-source fyzikální engine, který poskytuje simulaci pohybu objektů ve 2D prostoru. Je často využíván ve vývoji počítačových her ale také simulací a umožňuje snadné zpracování kolizí, gravitace, tuhosti objektů a dalších fyzikálních jevů. Tento engine používáme dostupnými knihovnami v Pythonu, ale původně byl implementován v~jazyce C++, což snižuje výpočetní náročnost simulace umožňuje nám iterovat přes rozsáhlé množství simulací. Pro tento engine jsme se rozhodli, jelikož z dostupných zdrojů víme, že stejný engine použili i vývojáři hry Poly Bridge \cite{Reddit}. 

\section{Aproximace hře Poly Bridge}

V naší simulaci jsme implementovali různé aspekty hry pomocí následujících komponent knihovny Box2D \cite{b2docs}:

\begin{itemize}
    \item \emph{Materiály}: Ty jsou modelovány jako dynamické objekty, pro které používáme \texttt{Box2D.b2DynamicBody}.
    \item \emph{Klouby}: Pro spoje různých materiálů jsme využili \texttt{Box2D.b2RevoluteJoint}.
    \item \emph{Zátež na prvky}: Abychom zjistili síly působící na jednotlivé elementy v simulaci, používáme metodu \texttt{b2body.GetReactionForce()}, která vrací reakční sílu vzniklou v důsledku interakce těles.
\end{itemize}

\subsection{Testy}

Abychom v naší simulaci co nejvěrněji napodobili chování fyzikálních prvků jako ve hře, zavedli jsme šest různých testů. Tyto testy zkoumají aspekty fyzikální simulace, jako jsou odolnost materiálů v proměnlivých podmínkách, hmotnost materiálů a interakci sil mezi objekty.

Zvolené testy jsou následující:

\begin{enumerate}
    \item \textbf{2 vozovky mezi dvěma pevnými body} Očekávaným výsledkem je, že konstrukce praskne pod zatížením samotných vozovek.
    \item \textbf{6 dřevěných dílů mezi dvěma pevnými body} Očekáváme, že konstrukce vydrží bez prasknutí.
    \item \textbf{7 dřevěných dílů mezi dvěma pevnými body} V tomto testu očekáváme, že konstrukce pod tíhou praskne.
    \item \textbf{Symetrický obrazec z 13.66 metrů vozovky, zavěšený na jednom kusu vozovky} Testujeme, zda vozovka unese zatížení bez prasknutí.
    \item \textbf{Symetrický obrazec z 14.66 metrů vozovky, zavěšený na jednom kusu vozovky} V tomto případě testujeme, zda konstrukce nevydrží zatížení a praskne.
    \item \textbf{Komplexní most z vozovek a dřeva, po kterém přejede auto} Cílem tohoto testu je prozkoumat interakci sil mezi různými materiály, kdy očekáváme, že most vydrží přejetí auta.
\end{enumerate}

Vizualizaci testů můžeme vidět na obrázku \ref{impl-fig:1}

\begin{figure}[ht]
    \centering
    \begin{minipage}{0.49\textwidth}
        \centering
        \includegraphics[width=\linewidth]{img/poly_tests.png}
    \end{minipage}\hfill
    \begin{minipage}{0.49\textwidth}
        \centering
        \includegraphics[width=\linewidth]{img/sim_tests.png}
    \end{minipage}
    \caption{Vizualizace testů ve hře polybridge (vlevo) a v simulaci (vpravo). Test č.1 nahoře uprostřed. Test č. 2 a 3 pod test č.1. Nalevo test č.4. Napravo test č.5. Most z 6. testu uprostřed. Vozidlo jede z levého břehu na pravý}
    \label{impl-fig:1}
\end{figure}

Naše implementace simulace ma $4$ různé parametry.
\begin{itemize}
    \item \emph{Hustota dřeva}: Hustota materiálů pro dřevo (\texttt{Box2D.b2Density}).
    \item \emph{Koeficient hustoty}: Násobek hustoty dřeva, který bude použit pro hustotu vozovky.
    \item \emph{Max. zatížení dřeva}: Maximální zatížení dřeva.
    \item \emph{Keoficient zatížení}: Násobek maximálního zatížení dřeva, který bude použit pro maximální zatížení vozovky.
\end{itemize}

Pro nalezení takových parametrů, které splní nejvíce testů jsme zvolili \textit{Random search} \cite{Random}. Prohledaná rozmezí těchto parametrů a jejich nejlepší nalezené hodnoty můžeme vidět v tabulce \ref{tab:1}. Bohužel se nám nepodařilo najít takové parametry, abychom splnili všechny. Rozhodli jsme se že $5.$ test vyřadíme.


\begin{table}[b!]
\centering
\begin{tabular}{l@{\hspace{1.5cm}}D{.}{,}{3}D{.}{,}{3}}
\toprule
\textbf{Parametr} & \multicolumn{1}{c}{\textbf{Rozmezí}} & \multicolumn{1}{c}{\textbf{Nalezená hodnota}$^a$} \\
\midrule
\emph{Hustota dřeva} & [0,01-3,01] & 1,348 \\
\emph{Max. zatížení dřeva} & [50,0-2050,0] & 765,0 \\
\emph{Koeficient hustoty} & [0,3-9,3] & 3,992 \\
\emph{Koeficient zatížení} & [0,1-5,1] & 0,821 \\
\bottomrule
\end{tabular}
\caption{Parametry pro simulaci, prohledávaná rozmezí a jejich nejlepší nalezená hodnoty.}\label{tab:1}
\footnotesize \textit{Pozn: $^a$Zaokrouhleno na 3 desetinná místa}
\end{table}


\subsection{Úrovně}

Jako testovací prostředí pro evoluční algoritmus jsme zvolili první $4$ úrovně z původní hry (viz obrázky \ref{impl-fig:2}, \ref{impl-fig:3}, \ref{impl-fig:4} a \ref{impl-fig:5}). Více úrovní jsme kvůli jejich komplexitě nezahrnuli.

\begin{figure}[ht]
    \centering
    \begin{minipage}{0.49\textwidth}
        \centering
        \includegraphics[width=\linewidth]{img/poly_lvl1.png}
    \end{minipage}\hfill
    \begin{minipage}{0.49\textwidth}
        \centering
        \includegraphics[width=\linewidth]{img/impl_lvl1.png}
    \end{minipage}
    \caption{Vizualizace $1.$ úrovně ve hře polybridge (vlevo) a v simulaci (vpravo)}
    \label{impl-fig:2}
\end{figure}

\begin{figure}[ht]
    \centering
    \begin{minipage}{0.49\textwidth}
        \centering
        \includegraphics[width=\linewidth]{img/poly_lvl2.png}
    \end{minipage}\hfill
    \begin{minipage}{0.49\textwidth}
        \centering
        \includegraphics[width=\linewidth]{img/impl_lvl2.png}
    \end{minipage}
    \caption{Vizualizace $2.$ úrovně ve hře polybridge (vlevo) a v simulaci (vpravo)}
    \label{impl-fig:3}
\end{figure}

\begin{figure}[ht]
    \centering
    \begin{minipage}{0.49\textwidth}
        \centering
        \includegraphics[width=\linewidth]{img/poly_lvl3.png}
    \end{minipage}\hfill
    \begin{minipage}{0.49\textwidth}
        \centering
        \includegraphics[width=\linewidth]{img/impl_lvl3.png}
    \end{minipage}
    \caption{Vizualizace $3.$ úrovně ve hře polybridge (vlevo) a v simulaci (vpravo)}
    \label{impl-fig:4}
\end{figure}

\begin{figure}[ht]
    \centering
    \begin{minipage}{0.49\textwidth}
        \centering
        \includegraphics[width=\linewidth]{img/poly_lvl4.png}
    \end{minipage}\hfill
    \begin{minipage}{0.49\textwidth}
        \centering
        \includegraphics[width=\linewidth]{img/impl_lvl4.png}
    \end{minipage}
    \caption{Vizualizace $4.$ úrovně ve hře polybridge (vlevo) a v simulaci (vpravo)}
    \label{impl-fig:5}
\end{figure}


\section{Aplikace evolučních algoritmů}

V následující sekci ukážeme, jak jsme navrhli různé typy genetický operátorů. Budeme používat následující značení.
\begin{itemize}
    \item $l_{max}$ je maximální délka materiálu
    \item $T$ je množina všech materiálů, které můžeme použít (vozovka, dřevo, nic)
    \item $g$ je genom jedince, nebo také jeden konkrétní most
    \item $d_{min}(g)$ minimální vzdálenost vozidla od úrovní definovaného bodu na druhé straně řeky, které se podařilo dosáhnout v simulaci
    \item $cost(g)$ cena mostu $g$
\end{itemize}

Ve všech případech se snažíme optimalizovat dvě hodnoty a to jak daleko naše vozidlo dojelo a cenu mostu. V rámci algoritmu se tedy primárně snažíme maximalozovat $-d_{max}(g)$ a sekundárně $-cost(g)$.

Naše fitness funkce $f$ bude tedy udávaná dvojicí čísel. $$f(g) = (-d_{min}(g), -cost(g))$$

\subsection{Jednoduchý návrh}

Nejjednodušší návrh danému problému by mohl vypadat následovně. Reprezentace genu je vektor dvojic čísel $c \in \{([0, \dots, x_{max}] \times [0, \dots, y_{max}])\}^n$ kde $x_{max} \in \N$ a $y_{max} \in \N$ je šířka a výška úrovně a vektor $t \in T^n$. Most se pak z genomu postavíme následovně. Iterujeme přes všechny dvojice z $c$ a zároveň i přes materiály z $t$. Mezi bod tvořený současnou dvojicí a posledním bodem, na který jsme materiál přidali, se snažíme položit současný materiál. Pokud je současný bod od toho minulého příliš daleko tak jej přiblížíme aby jeho vzdálenost byla $l_{max}$. Na začátku jako poslední bod vybereme nejvyšší kotvu na levém břehu. Tímto způsobem se snažíme napodobit posloupnost kliknutí, které by hráč normálně provedl.

Jake mutaci jsme zvolili náhodné posunutí pozice kliknutí (bodu) o $\pm 1$ s pravděpodobností $\frac{1}{n}$ a náhodnou změnu materiálu s pravděpodobností $\frac{1}{n}$.

Jako křížení jsme použili jednobodové křížení vektoru $c$ a $t$ podle stejně zvoleného náhodného bodu. Jako selekci jsme zvolili turnajovou selekci.

Jak můžeme vidět v experimentu \ref{exp:2}, algoritmu se nedaří stavět příliš kvalitní mosty. Domníváme se, že by tomu tak je z následujících důvodů.

\begin{itemize}
    \item Z principu reprezentace jedince je nepravděpodobné, aby vznikaly krátké hrany, které mohou být klíčové pro kvalitní řešení.
    \item I malá mutace na začátku genu může mít velký vliv na celkovou strukturu mostu.
    \item Křížení v naší reprezentaci nedává smysl.
    \item Fitness funkce nevrací dobrou zpětnou vazbu o kvalitě jedince (viz. obrázek \ref{impl-fig:6})
\end{itemize}

\subsection{Polární kódování}

Kvůli tomu, jak jsme reprezentovali jedince v předchozím návrhu, je nepravděpodobné, že budou vznikat krátké hrany, které mohou být zásadní pro dobré řešení. To, že náhodně zvolíme blízko od od posledního kliknutí je méně pravděpodobné, než že zvolíme bod daleko. Proto jsme navrhli kódování genu, kde dvojce z vektoru $c$ představují délku a úhel přidaného materiálu $c \in \{([0, l_{max}] \times [0, 2 \pi])\}^n$. Vektor $t$ zůstává stejný jako v předchozím případě.

Jake mutaci jsme zvolili přepsání hodnoty z $c$ na novou náhodně zvolenou s pravděpodobností $\frac{1}{n}$ a náhodnou změnu materiálu s pravděpodobností $\frac{1}{n}$.

Křížení a selekci použijeme stejnou jako v předchozím případě.

\subsection{Vylepšená fitness funkce}

Jedním z problémů, se kterým se náš současný návrh potýká je ten, že naše fitness funkce moc dobře nerozlišuje, jak je dané řešení kvalitní. Na obrázku \ref{impl-fig:6} můžeme vidět dva různé jednice, kteří mají stejnou fitness, a v kvalitě se značně liší.


\begin{figure}[ht]
    \centering
    \begin{minipage}{0.49\textwidth}
        \centering
       \includegraphics[width=\linewidth]{img/almost_good_bridge.png}
    \end{minipage}\hfill
    \begin{minipage}{0.49\textwidth}
        \centering
        \includegraphics[width=\linewidth]{img/bad_bridge.png}
    \end{minipage}
    \caption{Dva mosty se stejnou fitness, ale rozdílných kvalit}
    \label{impl-fig:6}
\end{figure}

Navrhli jsme proto dvě různé penalizace, které můžeme do fitness zapojit.

\begin{itemize}
    \item Penalizace za umisťování materiálů, který se nespojí s další materiálem. Lépe propojený most by měl mít lepší stabilitu.
    \item Penalizace za všechny kotvy, které jedinec nepoužil. V praxi použijem vzdálenost každého kliknutí ke všem nepoužitým kotvám. Most který používá více kotev by měl být stabilnější.
\end{itemize}

Nová fitness funkce pak vypadá následovně. $$f(g) = (-d_{min}(g) + \alpha \cdot mat + \beta \cdot anch, -cost(g))$$ Koeficinety $\alpha, \beta \in \R$ značí váhu penalizace a $mat, anch$ jsou hodnoty penalizace za nespojený materiál a nevyužité kotvy.

Použijeme stejnou selekci, mutaci a reprezentaci jedince jako v předchozím návrhu.

\subsection{Měnící se fitness}

V rámci našeho přístupu jsme narazili na specifický problém spojený se stabilitou mostu. Spočívá v tom, že dokud není most zcela dokončen, nedosahuje potřebné stability, které je potřeba pro přejetí vozidlem. Abychom se s tímto omezením vypořádali, rozhodli jsme se pro zjednodušení problému a využití techniky zvané \emph{inkrementální evoluční alogritmus} navrhnuté ve článku Mansouryho Nashaata et al. \cite{IGA}. Na začátku experimentu proto začínáme s břehy blíže umístěnými k sobě a jakmile dosáhneme dostatečně nízké průměrné fitness v celé populaci, vzdálenost postupně zvětšujeme, dokud nedosáhneme vzdálenosti definové úrovní. 

Použijeme stejnou selekci, mutaci a reprezentaci jedince jako v předchozím návrhu.

\subsection{Grafové kódování}

V této části bychom chtěli představit odlišný způsob, jak kódovat jednotlivce. Naše dosavadní kódování značně trpí tím, že je nepravděpodobné aby, se v jednom bodě spojilo více, než dva kusy materiálu. Tento problém se pokusíme vyřešit tím, že jedince budeme kódovat jako graf, tedy pomocí vrcholů a hran. To v praxi znamená, že gen jedince se skládá z množiny vrcholů $V$, množiny hran $E \subseteq V \times V$, funkce $\sigma_v : V \rightarrow \R^2$ která je projekcí $V$ do roviny a $\sigma_e : E \rightarrow T$. Naše navržené genetické operátory vypadají následovně.

\begin{itemize}
    \item \textbf{inicializace}: Do $V$ přidáme všechny kotvy z úrovně. Náhodně vybereme materiál $t \in T$, vrchol $v_1 \in V$, úhel $\varphi$ a délku $0 < l < l_{max}$. Vytvoříme nový vrchol $v_2$ a vložíme jej do $V$. Upravíme $\sigma_v$ tak, že $v_2$ se promítne na bod ve vzdálenosti $l$ a pod úhlem $\varphi$ od $\sigma_v(v_1)$. Vytvoříme novou hranu $(v_1, v_2)$ a přidáme jí do $E$ a zároveň upravíme $\sigma_e$ tak, že $\sigma_e((v_1, v_2)) = t$. Opakujeme dokud nevytvoříme $n$ nových vrcholů. Následně náhodně volíme dva vrcholy $v_1, v_2 \in V$. Pokud $||\sigma_v(v_1) - \sigma_v(v_2)|| < l_{max}$ vytvoříme novou hranu $(v_1, v_2)$ a přidáme do $E$. Upravíme $\sigma_e((v_1, v_2)) = t$ pro náhodně zvolené $t \in T$. Opakujeme $2n$-krát.
    \item \textbf{mutace}: Budeme rozlišovat mutaci pro vrcholy a mutaci pro hrany. Mutace pro vrcholy upraví $\sigma_v$ tak, že projekci vrcholu $v \in V$ přemístí na náhodně zvolený bod z $\{ x \in \R^2 | \forall v_2 \in Adj(v_1), ||x - \sigma_v(v_2)|| < l_{max}\}$ kde $Adj(v_1) = \{v \in V | (v_1, v) \in E\}$. Jinými slovy náhodně posumeme vrchol $v$ tak, aby nebyl příliš daleko od žádného vrcholu s nímž byl $v$ spojen hranou. Mutace pro hrany může hranu přidat, odebrat jí nebo změnit $\sigma_e$ náhodné hrany na jiný typ materiálu.
    \item \textbf{křížení}: Dva jedince můžeme skřížit následovně. Nechť $V_1, E_1$ je množina všech vrcholů a hran prvního z rodičů a $V_2, E_2$ druhého. Nechť $\sigma_{v_p}$ je spojení $\sigma_v$ funkcí obou rodičů a $\sigma_{e_p}$ spojení $\sigma_e$ funkcí obou rodičů. Zvolíme náhodně hranici $min\{\sigma_{v_p}(v)_x | v \in V_1 \cup V_2\} < p_x < max\{\sigma_{v_p}(v)_x | v \in V_1 \cup V_2\}$. Nechť $L = \{ v | v \in V_1, \sigma_{v_p}(v)_1 < p_x\}$ a $R = \{ v | v \in V_2, \sigma_{v_p}(v)_1 > p_x\}$. Vrcholy genu potomka pak budou z $V_p = R + L$ a hrany $E_p = (E_1 + E_2) \cap V_p \times V_p$, kterým navíc přidáme všechny $(v_1, v_2), v_1 \in R, v_2 \in L, ||\sigma_{v_p}(v_1) - \sigma_{v_p}(v_2)|| < l_{max}$ s pravděpodobností $\alpha$. $\sigma_v$ potomka bude $\sigma_{v_p}$ a stejně tak pro $\sigma_e$. Příklad takové mutace můžeme vidět na obrázku \ref{impl-fig:8}
\end{itemize}

\begin{figure}[ht]
    \centering
    \begin{minipage}{0.24\textwidth}
        \centering
        \includegraphics[width=\linewidth]{img/bridge_p1.png}
    \end{minipage}\hfill
    \begin{minipage}{0.24\textwidth}
        \centering
        \includegraphics[width=\linewidth]{img/bridge_p2.png}
    \end{minipage}
    \begin{minipage}{0.24\textwidth}
        \centering
        \includegraphics[width=\linewidth]{img/bridge_crossed.png}
    \end{minipage}
    \begin{minipage}{0.24\textwidth}
        \centering
        \includegraphics[width=\linewidth]{img/bridge_sups.png}
    \end{minipage}
    \caption{Příklad křížení dvou rodičů. Na 1. a 2. obrázku zleva jsou rodiče. Hranice pro náhodné křížení je vyznačena červenou čárou. Na 2. obrázku zprava křížení částí mosty. Na nejpravějším obrazku překžížení s vystužením}
    \label{impl-fig:8}
\end{figure}

Jako selekci použijeme turnajovou selekci.

\subsection{Lepší inicializace}

Do našeho algoritmu můžeme ještě zahrnout jednu z nejsilnějších techni pro evoluční algoritmy a to použití \textit{domain-specific} znalostí \cite{PASSONE2006192}. I když ještě nevíme, jak bude optimální most vypadat, dokážeme obecným způsobem navrhnout most, který sice nebude optimální, ale bude lepší, než náhodně umístěný materiál. Z tohoto důvodu jsme implementovali postup podobný tomu, který navrhli Hugo Lispector ve své diplomové práci \cite{Lispector2022}. Tento postup se skládá ze tří kroků.

\begin{enumerate}
    \item \textbf{Vytvoření vozovky}: Nejprve pro vozidlo vytvoříme vozovku a to tím způsobem, spojíme levý a pravý břeh vozovkami a náhodné délce.
    \item \textbf{Vytvoření opor}: Následně pro každou ještě nevyužitou kotvu vyberem jeden spojový kloub z předchozího kroku a spojíme je dřevěnými díly o náhodné délce.
    \item \textbf{Zpevnění}: Nakonec pro každý přidaný materiál náhodně vybereme nový bod tak, abychom jej mohli spojit jedním dílem dřeva se začátkem a koncem tohoto materiálu a navíc bod spojíme se všemi ostatními bodu v blízkém okolí s pravděpodobností $\omega$.
\end{enumerate}

Vizualizaci těchto tří kroků můžeme vidět na obrázku \ref{impl-fig:7}

\begin{figure}[ht]
    \centering
    \begin{minipage}{0.32\textwidth}
        \centering
        \includegraphics[width=\linewidth]{img/better_init1.png}
    \end{minipage}\hfill
    \begin{minipage}{0.32\textwidth}
        \centering
        \includegraphics[width=\linewidth]{img/better_init2.png}
    \end{minipage}
    \begin{minipage}{0.32\textwidth}
        \centering
        \includegraphics[width=\linewidth]{img/better_init3.png}
    \end{minipage}
    \caption{Výtváření jedince pomocí lepší inicializace. Krok \textbf{Vytvoření vozovky} vlevo, krok \textbf{Vytoření opor} uprostřed a krok \textbf{Zpevění} s $\omega = 1$ vpravo}
    \label{impl-fig:7}
\end{figure}

Použijeme selekci, křížení a mutaci z přechozího návrhu.
